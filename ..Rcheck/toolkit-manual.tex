\nonstopmode{}
\documentclass[a4paper]{book}
\usepackage[times,inconsolata,hyper]{Rd}
\usepackage{makeidx}
\makeatletter\@ifl@t@r\fmtversion{2018/04/01}{}{\usepackage[utf8]{inputenc}}\makeatother
% \usepackage{graphicx} % @USE GRAPHICX@
\makeindex{}
\begin{document}
\chapter*{}
\begin{center}
{\textbf{\huge Package `toolkit'}}
\par\bigskip{\large \today}
\end{center}
\ifthenelse{\boolean{Rd@use@hyper}}{\hypersetup{pdftitle = {toolkit: My Tool Function for R and Rmarkdown}}}{}
\ifthenelse{\boolean{Rd@use@hyper}}{\hypersetup{pdfauthor = {Guangyu Zhu}}}{}
\begin{description}
\raggedright{}
\item[Title]\AsIs{My Tool Function for R and Rmarkdown}
\item[Version]\AsIs{1.0.4}
\item[Author]\AsIs{Guangyu Zhu }\email{guangyuzhu@uri.edu}\AsIs{}
\item[Maintainer]\AsIs{Guangyu Zhu }\email{guangyuzhu@uri.edu}\AsIs{}
\item[Description]\AsIs{A collection of functions for enhancing R and Rmarkdown workflows.}
\item[License]\AsIs{MIT + file LICENSE}
\item[Encoding]\AsIs{UTF-8}
\item[Language]\AsIs{es}
\item[Roxygen]\AsIs{list(markdown = TRUE)}
\item[RoxygenNote]\AsIs{7.3.2}
\item[Suggests]\AsIs{testthat (>= 3.0.0)}
\item[Config/testthat/edition]\AsIs{3}
\item[Imports]\AsIs{clipr, dplyr, tidyr, flextable, ggplot2, kableExtra, tinytex,
knitr, magrittr, pacman, readxl, stringr, tibble, formattable,
readr, patchwork}
\item[URL]\AsIs{}\url{http://guang-yu-zhu.github.io/toolkit}\AsIs{}
\item[BugReports]\AsIs{}\url{https://github.com/guang-yu-zhu/toolkit/issues}\AsIs{}
\end{description}
\Rdcontents{Contents}
\HeaderA{\Rpercent{}>\Rpercent{}}{Pipe operator}{.Rpcent.>.Rpcent.}
\keyword{internal}{\Rpercent{}>\Rpercent{}}
%
\begin{Description}
See \code{magrittr::\LinkA{\Rpercent{}>\Rpercent{}}{.Rpcent.>.Rpcent.}} for details.
\end{Description}
%
\begin{Usage}
\begin{verbatim}
lhs %>% rhs
\end{verbatim}
\end{Usage}
%
\begin{Arguments}
\begin{ldescription}
\item[\code{lhs}] A value or the magrittr placeholder.

\item[\code{rhs}] A function call using the magrittr semantics.
\end{ldescription}
\end{Arguments}
%
\begin{Value}
The result of calling \code{rhs(lhs)}.
\end{Value}
\HeaderA{cleanLatex}{Remove LaTeX Auxiliary Files}{cleanLatex}
%
\begin{Description}
This function deletes auxiliary files generated after compiling LaTeX in the current working directory.
Common auxiliary file extensions such as \code{.log}, \code{.aux}, \code{.out}, etc., are removed.
\end{Description}
%
\begin{Usage}
\begin{verbatim}
cleanLatex(folder = getwd(), verbose = FALSE)
\end{verbatim}
\end{Usage}
%
\begin{Arguments}
\begin{ldescription}
\item[\code{folder}] The directory to clean. If not specified, it defaults to the current working directory (\code{getwd()}).

\item[\code{verbose}] Logical; if \code{TRUE}, prints the names of files that are deleted. Default is \code{FALSE}.
\end{ldescription}
\end{Arguments}
%
\begin{Details}
The function automatically removes common temporary files generated during the LaTeX compilation process,
including \code{.log}, \code{.aux}, \code{.out}, \code{.bbl}, \code{.toc}, \code{.bak}, and others. It does not prompt the user and will
clean the current directory without confirmation.
\end{Details}
%
\begin{Value}
The function returns \code{TRUE} if the files were successfully removed, and \code{FALSE} otherwise.
\end{Value}
%
\begin{Examples}
\begin{ExampleCode}
## Not run: 
# Clean the current working directory
cleanLatex()

# Clean a specific directory
cleanLatex("/path/to/latex/files", verbose = TRUE)

## End(Not run)
\end{ExampleCode}
\end{Examples}
\HeaderA{clip}{Capture output and copy to clipboard}{clip}
%
\begin{Description}
This function captures the output of an R object and writes it to a file, then copies the contents of the file to the clipboard.
\end{Description}
%
\begin{Usage}
\begin{verbatim}
clip(obj, file = "output.txt")
\end{verbatim}
\end{Usage}
%
\begin{Arguments}
\begin{ldescription}
\item[\code{obj}] The R object whose output is to be captured.

\item[\code{file}] The file to which the output will be written (default: "output.txt").
\end{ldescription}
\end{Arguments}
%
\begin{Value}
None
\end{Value}
\HeaderA{combine\_table}{Combine Two Tables with different statistics.}{combine.Rul.table}
%
\begin{Description}
This function combines two tables: one with raw counts and another with percentages (or other calculated values). It returns a formatted table that merges these values in a "count (percentage)" format.
\end{Description}
%
\begin{Usage}
\begin{verbatim}
combine_table(tab1, tab2, percent = TRUE)
\end{verbatim}
\end{Usage}
%
\begin{Arguments}
\begin{ldescription}
\item[\code{tab1}] A data frame or matrix of the first statistics.

\item[\code{tab2}] A data frame or matrix of the second statistics to be combined with \code{tab1}.

\item[\code{percent}] Logical. If \code{TRUE}, formats \code{tab2} as percentages (i.e., appends a \AsIs{\texttt{\%}} sign). If \code{FALSE}, combines the counts and values without the percentage sign.
\end{ldescription}
\end{Arguments}
%
\begin{Details}
This function takes two tables, converts them to data frames, and then combines their values in a "count (percentage)" format. If the \code{percent} argument is \code{TRUE}, the values in \code{tab2} are treated as percentages; otherwise, they are combined as raw values.
\end{Details}
%
\begin{Value}
A data frame with the combined counts and percentages (or values).
\end{Value}
%
\begin{Examples}
\begin{ExampleCode}
tab1<-margin.table(Titanic,c(1,4))
tab2 <- round(prop.table(tab1, margin = 1) * 100, 2)
combine_table(tab1,tab2,)

\end{ExampleCode}
\end{Examples}
\HeaderA{commit\_and\_tag}{Commit Changes and Tag Version in Git}{commit.Rul.and.Rul.tag}
%
\begin{Description}
This function commits all changes in the current directory to Git, creates a
tag for the specified version, and pushes both the commit and the tag to the
remote repository.
\end{Description}
%
\begin{Usage}
\begin{verbatim}
commit_and_tag(version = "1.0.0", message = paste("Release version", version))
\end{verbatim}
\end{Usage}
%
\begin{Arguments}
\begin{ldescription}
\item[\code{version}] A string specifying the version number for the tag. Default is "1.0.0".

\item[\code{message}] A string specifying the commit message. Default is constructed as "Release version " followed by the version number.
\end{ldescription}
\end{Arguments}
%
\begin{Value}
This function returns a message indicating whether the operations were successful or not.
\end{Value}
%
\begin{Examples}
\begin{ExampleCode}
## Not run: 
  commit_and_tag("1.0.2")

## End(Not run)

\end{ExampleCode}
\end{Examples}
\HeaderA{compare\_categorical}{Perform Cross-Tabulation and Fisher's Exact Test}{compare.Rul.categorical}
%
\begin{Description}
This function performs a cross-tabulation between two categorical variables and conducts Fisher's exact test, returning a formatted table with counts, proportions, and p-values.
\end{Description}
%
\begin{Usage}
\begin{verbatim}
compare_categorical(
  dat,
  col_var,
  row_var,
  varname = "",
  total = FALSE,
  colname = "Variables"
)
\end{verbatim}
\end{Usage}
%
\begin{Arguments}
\begin{ldescription}
\item[\code{dat}] A data frame containing the variables.

\item[\code{col\_var}] The grouping variable (categorical or factor) to classify the data into two or more groups.

\item[\code{row\_var}] The categorical variable to compare against \code{col\_var}.

\item[\code{varname}] Optional. A character string to label the \code{row\_var} row in the output table (default is an empty string).

\item[\code{total}] Logical; if TRUE, includes an overall total in the table.

\item[\code{colname}] Optional. The name of the first column in the output table (default is 'Variables').
\end{ldescription}
\end{Arguments}
%
\begin{Details}
This function creates a contingency table for \code{col\_var} and \code{row\_var} using \code{xtabs}, adds row margins if requested, computes proportions, and performs Fisher's exact test to assess the association between the two variables. The output is a formatted table with counts, proportions, and the test's p-value.
\end{Details}
%
\begin{Value}
A data frame with the cross-tabulation results, including counts, proportions, and the p-value from Fisher's exact test.
\end{Value}
%
\begin{Examples}
\begin{ExampleCode}
compare_categorical(mtcars, 'cyl', 'gear', varname = 'Gear')

\end{ExampleCode}
\end{Examples}
\HeaderA{compare\_numerical}{Compute Mean and SD for Two Samples by Grouping Variable and Perform One-Way ANOVA}{compare.Rul.numerical}
%
\begin{Description}
This function calculates the mean and standard deviation of a numerical variable for each level of a grouping variable and performs a one-way ANOVA test. The output is a formatted table that includes group-wise means, standard deviations, and a p-value.
\end{Description}
%
\begin{Usage}
\begin{verbatim}
compare_numerical(
  dat,
  col_var,
  row_var,
  varname = "",
  total = FALSE,
  colname = "Variables"
)
\end{verbatim}
\end{Usage}
%
\begin{Arguments}
\begin{ldescription}
\item[\code{dat}] A data frame containing the variables.

\item[\code{col\_var}] The grouping variable (factor or categorical).

\item[\code{row\_var}] The numerical variable for which mean and standard deviation are computed.

\item[\code{varname}] A character string to label the row in the output table.

\item[\code{total}] Logical; if TRUE, includes overall mean and SD.

\item[\code{colname}] A character string for naming the output table column.
\end{ldescription}
\end{Arguments}
%
\begin{Details}
The function computes the mean and standard deviation for \code{row\_var} within each group defined by \code{col\_var}, as well as overall. It performs a one-way ANOVA test to assess differences in means between groups. Results are presented in a table that includes the p-value.
\end{Details}
%
\begin{Value}
A data frame containing group-wise means and standard deviations of \code{row\_var} for each level of \code{col\_var}, along with the p-value from the one-way ANOVA test.
\end{Value}
%
\begin{Examples}
\begin{ExampleCode}
compare_numerical(mtcars, 'cyl', 'mpg', varname = 'Miles per Gallon')

\end{ExampleCode}
\end{Examples}
\HeaderA{compile\_handouts}{Create and Compile Handouts from a List of TeX Files}{compile.Rul.handouts}
%
\begin{Description}
This function processes a list of LaTeX files, creates corresponding handout files using a specified template, compiles them into PDFs, and removes the intermediate handout files after successful compilation.
\end{Description}
%
\begin{Usage}
\begin{verbatim}
compile_handouts(file_list, template_path = "handout_tex/handout.tex")
\end{verbatim}
\end{Usage}
%
\begin{Arguments}
\begin{ldescription}
\item[\code{file\_list}] A list of TeX filenames to be processed and compiled into handouts.

\item[\code{template\_path}] A string specifying the path to the template LaTeX file (default is "handout\_tex/handout.tex").
\end{ldescription}
\end{Arguments}
%
\begin{Details}
The function reads a template LaTeX file and loops over a user-provided list of TeX files. For each TeX file, the function generates a corresponding handout file by replacing a placeholder in the template, compiles the file into a PDF using \code{tinytex::latexmk}, and then deletes the handout TeX file once the PDF is generated. If an error occurs during compilation, it skips to the next file.
\end{Details}
%
\begin{Value}
This function does not return a value but compiles PDFs and deletes the temporary handout TeX files after compilation.
\end{Value}
%
\begin{Examples}
\begin{ExampleCode}
## Not run: 
  tex_files <- list.files(pattern = "^Sta.*\\.tex")
  compile_handouts(tex_files)

## End(Not run)

\end{ExampleCode}
\end{Examples}
\HeaderA{compile\_rnws}{Knit Files to PDF and Track Status}{compile.Rul.rnws}
%
\begin{Description}
This function takes a list of Rnw or other LaTeX source files, attempts to knit each one to a PDF using the \code{knit2pdf} function, and tracks the success or failure of each attempt.
\end{Description}
%
\begin{Usage}
\begin{verbatim}
compile_rnws(file_list = NULL)
\end{verbatim}
\end{Usage}
%
\begin{Arguments}
\begin{ldescription}
\item[\code{file\_list}] A character vector containing the file paths to the Rnw or LaTeX source files. If NULL, the function will search for all \code{.Rnw} files in the current directory.
\end{ldescription}
\end{Arguments}
%
\begin{Details}
The function iterates through the list of files, attempts to compile each one using \code{knit2pdf}, and stores the result (success or failure) in a data frame. If an error occurs during the knitting process, the function will skip that file and proceed to the next one.
\end{Details}
%
\begin{Value}
A data frame with two columns:
\begin{description}

\item[file] The name of the file that was processed.
\item[success] Logical value indicating whether the knitting process was successful (\code{TRUE}) or encountered an error (\code{FALSE}).

\end{description}

\end{Value}
%
\begin{Examples}
\begin{ExampleCode}
## Not run: 
  file_list = list.files(pattern = "^Sta.*Rnw", ignore.case = TRUE)
  knit_results <- compile_rnws(file_list)
  print(knit_results)

## End(Not run)

\end{ExampleCode}
\end{Examples}
\HeaderA{compile\_texs}{Compile LaTeX Files}{compile.Rul.texs}
%
\begin{Description}
This function compiles a list of LaTeX (.tex) files using \code{tinytex::latexmk}.
If no file list is provided, it compiles all \code{.tex} files in the current working directory.
\end{Description}
%
\begin{Usage}
\begin{verbatim}
compile_texs(file_list = NULL, clean = TRUE)
\end{verbatim}
\end{Usage}
%
\begin{Arguments}
\begin{ldescription}
\item[\code{file\_list}] A character vector of LaTeX file names to be compiled. If NULL, all \code{.tex} files in the working directory will be compiled.

\item[\code{clean}] Logical indicating whether to clean auxiliary files after compilation (default is TRUE).
\end{ldescription}
\end{Arguments}
%
\begin{Value}
A data frame containing the names of the files and their compilation success status.
\end{Value}
%
\begin{Examples}
\begin{ExampleCode}
## Not run: 
  compile_latex_files()
  compile_latex_files(c("file1.tex", "file2.tex"))

## End(Not run)

\end{ExampleCode}
\end{Examples}
\HeaderA{count\_percent}{Calculate Counts and Percentages of Factor Levels}{count.Rul.percent}
%
\begin{Description}
This function computes the counts and percentages for each level of a factor or categorical variable.
It returns a character vector with each level's count and its percentage of the total, formatted to a specified number of decimal places.
\end{Description}
%
\begin{Usage}
\begin{verbatim}
count_percent(x, digits = 2)
\end{verbatim}
\end{Usage}
%
\begin{Arguments}
\begin{ldescription}
\item[\code{x}] A factor or a categorical variable.

\item[\code{digits}] An integer indicating the number of decimal places to which the
percentages will be rounded. Defaults to 2.
\end{ldescription}
\end{Arguments}
%
\begin{Value}
A character vector where each element corresponds to a level of \code{x}.
Each element is formatted as 'count (percentage\%)', where 'count' is the number of occurrences of the level
and 'percentage' is the percentage of the total, rounded to \code{digits} decimal places.
\end{Value}
%
\begin{Examples}
\begin{ExampleCode}
levels= LETTERS[1:3]
factor1 <- levels%>%sample(size = 10, replace = TRUE)%>%factor()
count_percent(factor1,1)

\end{ExampleCode}
\end{Examples}
\HeaderA{cut\_txt}{Cut text file into chunks with specified number of chapters}{cut.Rul.txt}
%
\begin{Description}
This function reads a text file and splits it into chunks where each chunk contains a specified number of chapters.
Chapters are identified by markers like "第.*章".
\end{Description}
%
\begin{Usage}
\begin{verbatim}
cut_txt(file, marker = "第.*章", num_chapters = 25)
\end{verbatim}
\end{Usage}
%
\begin{Arguments}
\begin{ldescription}
\item[\code{file}] The path to the text file to be cut.

\item[\code{marker}] The regular expression pattern to identify chapters in the text file.

\item[\code{num\_chapters}] The number of chapters to include in each chunk. Default is 25.
\end{ldescription}
\end{Arguments}
%
\begin{Value}
NULL (The function writes output files.)
\end{Value}
%
\begin{Examples}
\begin{ExampleCode}
## Not run: 
cut_txt(file = "your_file.txt", marker = "第.*章", num_chapters = 25)

## End(Not run)

\end{ExampleCode}
\end{Examples}
\HeaderA{draw\_normal}{Draw a normal distribution plot using ggplot2}{draw.Rul.normal}
%
\begin{Description}
This function creates a ggplot2 plot representing the probability density function (PDF)
of a normal distribution with a specified mean and standard deviation.
\end{Description}
%
\begin{Usage}
\begin{verbatim}
draw_normal(mu = 0, sd = 1, color_density = "blue")
\end{verbatim}
\end{Usage}
%
\begin{Arguments}
\begin{ldescription}
\item[\code{mu}] Mean of the normal distribution. Default is 0.

\item[\code{sd}] Standard deviation of the normal distribution. Default is 1.

\item[\code{color\_density}] Color of the density line. Default is "blue".
\end{ldescription}
\end{Arguments}
%
\begin{Value}
A ggplot2 object representing the normal distribution plot.
\end{Value}
%
\begin{Examples}
\begin{ExampleCode}
draw_normal(mu = 0, sd = 1, color_density = "red")

\end{ExampleCode}
\end{Examples}
\HeaderA{fun\_links}{Generate Links to Function Documentation}{fun.Rul.links}
%
\begin{Description}
This function takes a list of function names and generates a markdown
formatted string with links to their documentation.
\end{Description}
%
\begin{Usage}
\begin{verbatim}
fun_links(function_names, package_name = "toolkit")
\end{verbatim}
\end{Usage}
%
\begin{Arguments}
\begin{ldescription}
\item[\code{function\_names}] A character vector of function names.

\item[\code{package\_name}] A string specifying the name of the package (default is "toolkit").
\end{ldescription}
\end{Arguments}
%
\begin{Value}
A single concatenated string containing markdown links to the function documentation.
\end{Value}
%
\begin{Examples}
\begin{ExampleCode}
function_list <- c("compile_rnws", "cleanFolder", "knit_rnws")
result <- fun_links(function_list, package_name = "toolkit")
cat(result)
\end{ExampleCode}
\end{Examples}
\HeaderA{meanSD}{Calculate Mean and Standard Deviation, and Format the Output}{meanSD}
%
\begin{Description}
This function computes the mean and standard deviation of a numeric vector
and returns a formatted string with the mean and standard deviation, rounded to
a specified number of decimal places.
\end{Description}
%
\begin{Usage}
\begin{verbatim}
meanSD(x, digits = 2)
\end{verbatim}
\end{Usage}
%
\begin{Arguments}
\begin{ldescription}
\item[\code{x}] A numeric vector for which the mean and standard deviation are to be computed.

\item[\code{digits}] An integer indicating the number of decimal places to which the
results will be rounded. Defaults to 2.
\end{ldescription}
\end{Arguments}
%
\begin{Value}
A character string containing the mean and standard deviation of \code{x},
formatted as 'mean (standard deviation)', each rounded to \code{digits} decimal places.
\end{Value}
%
\begin{Examples}
\begin{ExampleCode}
meanSD(c(10.5, 5.3, 7.8), digits = 3)
\end{ExampleCode}
\end{Examples}
\HeaderA{plot\_coef}{Plot Regression Coefficients}{plot.Rul.coef}
%
\begin{Description}
This function creates a bar plot to visualize regression coefficients, with optional ordering, intercept removal, and coefficient selection.
\end{Description}
%
\begin{Usage}
\begin{verbatim}
plot_coef(
  coefficients,
  remove_intercept = FALSE,
  order_coef = TRUE,
  select = NULL,
  title = "Regression Coefficients"
)
\end{verbatim}
\end{Usage}
%
\begin{Arguments}
\begin{ldescription}
\item[\code{coefficients}] A named vector or data frame of regression coefficients.

\item[\code{remove\_intercept}] Logical. If TRUE, the intercept (first row) will be removed from the plot (default: FALSE).

\item[\code{order\_coef}] Logical. If TRUE, the coefficients will be sorted by absolute value before plotting (default: TRUE).

\item[\code{select}] A logical vector to specify which coefficients are selected. If provided, the plot will highlight selected coefficients (default: NULL).

\item[\code{title}] A character string specifying the title of the plot (default: "Regression Coefficients").
\end{ldescription}
\end{Arguments}
%
\begin{Details}
This function takes regression coefficients and creates a horizontal bar plot. It allows for sorting by the absolute value of coefficients, removing the intercept, and highlighting selected coefficients if a logical vector is provided. The plot is useful for interpreting regression models.
\end{Details}
%
\begin{Value}
A ggplot object displaying the regression coefficients as a bar plot.
\end{Value}
%
\begin{Examples}
\begin{ExampleCode}
# Example usage:
fit<-lm(mpg~.,data=mtcars)
coef(fit)%>%plot_coef(remove_intercept = TRUE)

\end{ExampleCode}
\end{Examples}
\HeaderA{print\_table}{Print a Styled Table for HTML or LaTeX Output}{print.Rul.table}
%
\begin{Description}
This function prints a styled table in either HTML or LaTeX format, depending on the output document type.
\end{Description}
%
\begin{Usage}
\begin{verbatim}
print_table(
  df,
  num_col = 1,
  rowname = NA,
  caption = "",
  digits = 2,
  fontsize = 9,
  big.mark = ",",
  na_str = "",
  ...
)
\end{verbatim}
\end{Usage}
%
\begin{Arguments}
\begin{ldescription}
\item[\code{df}] The data frame to be formatted as a table.

\item[\code{num\_col}] An integer specifying the number of columns to split the output table into (default: 1).

\item[\code{rowname}] A character string specifying the column name for the row names; if not provided, the row names will not be printed (default: NA).

\item[\code{caption}] A character string specifying the caption for the table (default: '').

\item[\code{digits}] An integer indicating the number of digits to display (default: 2).

\item[\code{fontsize}] Font size for the table (default: 9).

\item[\code{big.mark}] A character string used as the thousands separator for numbers (default: ',').

\item[\code{na\_str}] A string used to replace missing values (default: '').

\item[\code{...}] Additional arguments to be passed to the formatting functions.
\end{ldescription}
\end{Arguments}
%
\begin{Value}
A styled table in HTML or LaTeX format, depending on the document type.
\end{Value}
%
\begin{Examples}
\begin{ExampleCode}
print_table(mtcars[1:10, 1:3], num_col = 2, rowname = 'car')

\end{ExampleCode}
\end{Examples}
\HeaderA{print\_table\_stripe}{Print a styled LaTeX table with alternating row colors}{print.Rul.table.Rul.stripe}
%
\begin{Description}
This function prints a LaTeX table using the kableExtra package with alternating row colors.
\end{Description}
%
\begin{Usage}
\begin{verbatim}
print_table_stripe(df, fontsize = 7, stripe_color = "cyan!15", ...)
\end{verbatim}
\end{Usage}
%
\begin{Arguments}
\begin{ldescription}
\item[\code{df}] The data frame to be formatted as a table.

\item[\code{fontsize}] Font size for the table (default: 7).

\item[\code{stripe\_color}] Color for the alternating row stripes (default: "cyan!15").

\item[\code{...}] Additional arguments to be passed to the kbl function.
\end{ldescription}
\end{Arguments}
%
\begin{Value}
A LaTeX table with alternating row colors.
\end{Value}
%
\begin{Examples}
\begin{ExampleCode}
print_table_stripe(mtcars[1:5, 1:3])

\end{ExampleCode}
\end{Examples}
\HeaderA{rmd\_hook\_set}{Set custom knit hooks for R Markdown}{rmd.Rul.hook.Rul.set}
%
\begin{Description}
This function sets custom knit hooks for R Markdown documents to customize behavior during knitting.
\end{Description}
%
\begin{Usage}
\begin{verbatim}
rmd_hook_set()
\end{verbatim}
\end{Usage}
%
\begin{Details}
The function sets inline knit hook to format numeric values and par hook to customize plotting parameters.
\end{Details}
%
\begin{Value}
None
\end{Value}
%
\begin{Examples}
\begin{ExampleCode}
rmd_hook_set()

\end{ExampleCode}
\end{Examples}
\HeaderA{rmd\_opts\_set}{Set default options for code chunks in R Markdown}{rmd.Rul.opts.Rul.set}
%
\begin{Description}
This function sets default options for code chunks in R Markdown documents based on the output format.
\end{Description}
%
\begin{Usage}
\begin{verbatim}
rmd_opts_set()
\end{verbatim}
\end{Usage}
%
\begin{Details}
The function determines the output format (LaTeX or HTML) and sets appropriate chunk options.
For LaTeX output, it sets options such as comment style, output width, figure settings, caching,
and code styling. For HTML output, it sets similar options tailored for HTML rendering.
\end{Details}
%
\begin{Value}
None
\end{Value}
%
\begin{Examples}
\begin{ExampleCode}
rmd_opts_set()

\end{ExampleCode}
\end{Examples}
\HeaderA{set\_ggplot\_theme}{Set custom theme for ggplot2 plots}{set.Rul.ggplot.Rul.theme}
%
\begin{Description}
This function sets a custom theme for ggplot2 plots with a light background color.
\end{Description}
%
\begin{Usage}
\begin{verbatim}
set_ggplot_theme()
\end{verbatim}
\end{Usage}
%
\begin{Details}
The function sets various graphical parameters for ggplot2 plots to customize their appearance.
\end{Details}
%
\begin{Value}
None
\end{Value}
%
\begin{Examples}
\begin{ExampleCode}
set_ggplot_theme()

\end{ExampleCode}
\end{Examples}
\HeaderA{sourceDir}{Source All R Scripts in a Directory}{sourceDir}
%
\begin{Description}
This function sources all R scripts from a specified directory. It recursively searches for files
with extensions \code{.R}, \code{.r}, \code{.S}, \code{.s}, \code{.Q}, and \code{.q} and sources them into the current environment.
\end{Description}
%
\begin{Usage}
\begin{verbatim}
sourceDir(path, trace = TRUE, ...)
\end{verbatim}
\end{Usage}
%
\begin{Arguments}
\begin{ldescription}
\item[\code{path}] A character string specifying the path to the directory containing the R scripts.

\item[\code{trace}] A logical value indicating whether to print the names of the files being sourced. Default is TRUE.

\item[\code{...}] Additional arguments to be passed to the \code{\LinkA{source}{source}} function.
\end{ldescription}
\end{Arguments}
%
\begin{Value}
Invisible NULL. This function is called for its side effects.
\end{Value}
\HeaderA{summarize\_plots}{Summarize and Plot Columns of a Data Frame}{summarize.Rul.plots}
%
\begin{Description}
This function generates summary plots for each column in a data frame.
For numeric columns, it produces histograms, and for factor or character columns,
it creates bar plots of the frequencies.
\end{Description}
%
\begin{Usage}
\begin{verbatim}
summarize_plots(df, ncol = 3, fill = "firebrick", width = 10, height = 4, ...)
\end{verbatim}
\end{Usage}
%
\begin{Arguments}
\begin{ldescription}
\item[\code{df}] A data frame to summarize and plot.

\item[\code{ncol}] An integer specifying the number of columns to arrange the plots in (default: 3).
This parameter is useful when displaying plots in a grid.

\item[\code{fill}] A string specifying the fill color for the bars in histograms or bar plots (default: "firebrick").

\item[\code{width}] Numeric value specifying the width of the plot (default: 10).

\item[\code{height}] Numeric value specifying the height of the plot (default: 4).

\item[\code{...}] Additional arguments passed to \code{patchwork::wrap\_plots()} for customizing the layout.
\end{ldescription}
\end{Arguments}
%
\begin{Details}
This function checks the data type of each column in the provided data frame.
\begin{itemize}

\item{} For numeric columns, it creates histograms showing the distribution of the values.
\item{} For categorical columns (factors or characters), it converts any character columns to factors
and generates bar plots showing the counts of each level.

\end{itemize}


The function returns a combined set of plots arranged in a grid using \code{patchwork::wrap\_plots}.
You can control the number of columns in the grid using the \code{ncol} argument.
\end{Details}
%
\begin{Value}
A combined patchwork of \code{ggplot2} objects arranged in a grid, where each plot corresponds
to one of the columns of the data frame.
\end{Value}
%
\begin{Examples}
\begin{ExampleCode}
# Example with the mtcars dataset
summarize_plots(mtcars)

\end{ExampleCode}
\end{Examples}
\HeaderA{summarize\_table}{Summarize Data Frame by Column Type}{summarize.Rul.table}
%
\begin{Description}
This function provides a summary of each column in a data frame, generating different statistics based on the type of data (numeric, factor, character, logical, or other).
It returns a tibble with the variable names and associated summary statistics, categorized by their data type.
\end{Description}
%
\begin{Usage}
\begin{verbatim}
summarize_table(df)
\end{verbatim}
\end{Usage}
%
\begin{Arguments}
\begin{ldescription}
\item[\code{df}] A data frame containing the variables to be summarized.
\end{ldescription}
\end{Arguments}
%
\begin{Value}
A tibble with summary statistics for each variable in the input data frame. The summary includes:
\begin{ldescription}
\item[\code{Type}] The data type of the variable (Numeric, Factor, Character, Logical, Other).
\item[\code{Mean}] The mean of the numeric variables (if applicable).
\item[\code{Median}] The median of the numeric variables (if applicable).
\item[\code{SD}] The standard deviation of the numeric variables (if applicable).
\item[\code{Min}] The minimum value of the numeric variables (if applicable).
\item[\code{Max}] The maximum value of the numeric variables (if applicable).
\item[\code{Levels}] The number of levels for factor variables (if applicable).
\item[\code{Unique}] The number of unique values for character or factor variables.
\item[\code{Most\_Common}] The most frequent value for character or factor variables.
\item[\code{Frequency}] The frequency of the most common value for character or factor variables.
\item[\code{True\_Count}] The number of TRUE values for logical variables.
\item[\code{False\_Count}] The number of FALSE values for logical variables.
\item[\code{NA\_Count}] The number of missing values for each variable.
\end{ldescription}
\end{Value}
%
\begin{Examples}
\begin{ExampleCode}
summarize_table(mtcars)

\end{ExampleCode}
\end{Examples}
\HeaderA{summary\_w2}{Summarize W2 Data for a Given Year}{summary.Rul.w2}
%
\begin{Description}
This function reads W2 data from an Excel file for a given year, calculates
the total amount for each person and category, and generates a summary table.
\end{Description}
%
\begin{Usage}
\begin{verbatim}
summary_w2(year, by_person = FALSE)
\end{verbatim}
\end{Usage}
%
\begin{Arguments}
\begin{ldescription}
\item[\code{year}] The year for which W2 data is to be summarized.

\item[\code{by\_person}] Logical indicating whether to summarize the data by person (default: FALSE).
\end{ldescription}
\end{Arguments}
%
\begin{Value}
A flextable object representing the summary table of W2 data for the specified year.
\end{Value}
\HeaderA{zkbl}{Format and style a table for LaTeX output}{zkbl}
%
\begin{Description}
This function formats and styles a table for LaTeX output using the kableExtra package.
\end{Description}
%
\begin{Usage}
\begin{verbatim}
zkbl(
  table,
  booktabs = TRUE,
  escape = TRUE,
  format.args = list(decimal.mark = ".", big.mark = ","),
  ...
)
\end{verbatim}
\end{Usage}
%
\begin{Arguments}
\begin{ldescription}
\item[\code{table}] The data table to be formatted.

\item[\code{booktabs}] Logical indicating whether to use the booktabs formatting (default: TRUE).

\item[\code{escape}] Logical indicating whether to escape special characters in the table (default: TRUE).

\item[\code{...}] Additional arguments to be passed to the kbl function.
\end{ldescription}
\end{Arguments}
%
\begin{Value}
A LaTeX-formatted table with specified styling.
\end{Value}
%
\begin{Examples}
\begin{ExampleCode}
zkbl(mtcars[1:5,])

\end{ExampleCode}
\end{Examples}
\HeaderA{zpvalue}{Format p-value for Display}{zpvalue}
%
\begin{Description}
This function formats a p-value for display, rounding it to three decimal places. If the p-value is smaller than 0.001, it returns '<0.001'.
\end{Description}
%
\begin{Usage}
\begin{verbatim}
zpvalue(x)
\end{verbatim}
\end{Usage}
%
\begin{Arguments}
\begin{ldescription}
\item[\code{x}] A numeric value representing a p-value.
\end{ldescription}
\end{Arguments}
%
\begin{Value}
A character string. If \code{x > 0.001}, it returns the p-value rounded to three decimal places. If \code{x <= 0.001}, it returns '<0.001'.
\end{Value}
%
\begin{Examples}
\begin{ExampleCode}
# Example usage of the function:
zpvalue(0.005)
zpvalue(0.0005)

\end{ExampleCode}
\end{Examples}
\HeaderA{zrange}{Calculate and Format Range of a Numeric Vector}{zrange}
%
\begin{Description}
This function calculates the range (minimum and maximum values) of a given numeric vector.
It then formats and rounds these values to a specified number of decimal places, returning them as a string.
\end{Description}
%
\begin{Usage}
\begin{verbatim}
zrange(x, digits = 2)
\end{verbatim}
\end{Usage}
%
\begin{Arguments}
\begin{ldescription}
\item[\code{x}] A numeric vector for which the range (min and max) is to be calculated.

\item[\code{digits}] An integer indicating the number of decimal places to which the
results will be rounded. Defaults to 2.
\end{ldescription}
\end{Arguments}
%
\begin{Value}
A character string representing the range of \code{x}, formatted as
'(min, max)', where 'min' and 'max' are the minimum and maximum values of \code{x}
rounded to \code{digits} decimal places.
\end{Value}
%
\begin{Examples}
\begin{ExampleCode}
zrange(c(1.234, 5.678, 9.012))
zrange(c(10, 20, 30, 40, 50), digits = 1)

\end{ExampleCode}
\end{Examples}
\HeaderA{zround}{Round and Format Numbers to Specific Decimal Places}{zround}
%
\begin{Description}
This function takes a numeric vector or a single numeric value and
formats it to a specified number of decimal places. The function
returns a character vector with the numbers formatted as strings.
\end{Description}
%
\begin{Usage}
\begin{verbatim}
zround(x, digits = 2, method = 1)
\end{verbatim}
\end{Usage}
%
\begin{Arguments}
\begin{ldescription}
\item[\code{x}] A numeric vector or a single numeric value.

\item[\code{digits}] An integer indicating the number of decimal places
to format the numbers. Defaults to 2.

\item[\code{method}] An integer (1 or 2) specifying the rounding method:
\begin{itemize}

\item{} \code{1}: Uses \code{round()} followed by \code{format()}.
\item{} \code{2}: Uses \code{sprintf()} to format the numbers.

\end{itemize}

\end{ldescription}
\end{Arguments}
%
\begin{Value}
A character vector with each number in \code{x} formatted to
\code{digits} decimal places.
\end{Value}
%
\begin{Examples}
\begin{ExampleCode}
x = c(5.555, 1.115, -0.002)
zround(x, method = 1)
zround(x, method = 2)
formatC(x, digits = 2, format = "f")
formattable::formattable(x, digits = 2, format = "f")

\end{ExampleCode}
\end{Examples}
\printindex{}
\end{document}
